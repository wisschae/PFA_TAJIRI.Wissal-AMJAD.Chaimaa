% ============================================================================
% INTRODUCTION GÉNÉRALE
% ============================================================================

\chapter*{Introduction Générale}
\addcontentsline{toc}{chapter}{Introduction Générale}
\markboth{Introduction Générale}{Introduction Générale}

La Coupe du Monde FIFA 2026, co-organisée par les États-Unis, le Mexique et le Canada, représente un défi majeur en matière de sécurisation des accès aux systèmes d'information. Face à la sophistication croissante des cybermenaces et à la diversité des profils utilisateurs (spectateurs, personnel, administrateurs), les mécanismes d'authentification traditionnels se révèlent insuffisants. La problématique centrale de ce projet est la suivante : comment concevoir un système de contrôle d'accès hybride combinant authentification multi-facteurs, reconnaissance biométrique et gestion dynamique des niveaux d'autorisation, tout en préservant une expérience utilisateur fluide ?

Pour répondre à ce besoin, nous développons la plateforme \textbf{FIFA World Cup 2026 – Unity Hub}, un système de reconnaissance hybride intégrant une authentification forte (mot de passe, OTP, reconnaissance faciale) et un contrôle d'accès granulaire à quatre niveaux d'habilitation. L'architecture repose sur une séparation claire entre frontend React, backend Spring Boot et microservice biométrique FastAPI, garantissant modularité et évolutivité.

Ce rapport s'articule en quatre chapitres : présentation du cadre de projet et étude de l'existant (chapitre 1), spécification des besoins fonctionnels et non fonctionnels (chapitre 2), conception UML et architecture du système (chapitre 3), et réalisation technique avec présentation des interfaces (chapitre 4).
