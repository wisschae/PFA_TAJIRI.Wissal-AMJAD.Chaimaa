% ============================================================================
% ANNEXES
% ============================================================================

\chapter*{Annexes}
\addcontentsline{toc}{chapter}{Annexes}
\markboth{Annexes}{Annexes}

Ce document annexe regroupe des éléments complémentaires au rapport principal : table des acronymes utilisés, extraits techniques illustratifs et récapitulatifs de référence.

% ============================================================================
% ANNEXE A1 : TABLE DES ACRONYMES
% ============================================================================

\section*{A1 – Table des acronymes}
\addcontentsline{toc}{section}{A1 – Table des acronymes}

Le tableau \ref{tab:acronymes} récapitule les principaux acronymes et abréviations utilisés tout au long de ce rapport.

\begin{table}[H]
\centering
\caption{Table des acronymes}
\label{tab:acronymes}
\begin{tabular}{|l|l|}
\hline
\textbf{Acronyme} & \textbf{Signification} \\
\hline
API & Application Programming Interface (Interface de programmation applicative) \\
\hline
CORS & Cross-Origin Resource Sharing (Partage de ressources entre origines) \\
\hline
CSRF & Cross-Site Request Forgery (Falsification de requête intersite) \\
\hline
JWT & JSON Web Token (Jeton Web JSON) \\
\hline
MFA & Multi-Factor Authentication (Authentification multi-facteurs) \\
\hline
OTP & One-Time Password (Mot de passe à usage unique) \\
\hline
RBAC & Role-Based Access Control (Contrôle d'accès basé sur les rôles) \\
\hline
REST & Representational State Transfer (Transfert d'état représentationnel) \\
\hline
RGPD & Règlement Général sur la Protection des Données \\
\hline
TLS & Transport Layer Security (Sécurité de la couche transport) \\
\hline
TOTP & Time-based One-Time Password (Mot de passe temporel à usage unique) \\
\hline
UML & Unified Modeling Language (Langage de modélisation unifié) \\
\hline
XSS & Cross-Site Scripting (Script intersite) \\
\hline
\end{tabular}
\end{table}

% ============================================================================
% ANNEXE A2 : EXTRAITS TECHNIQUES
% ============================================================================

\section*{A2 – Extraits techniques illustratifs}
\addcontentsline{toc}{section}{A2 – Extraits techniques illustratifs}

Cette section présente des extraits de code simplifiés illustrant les mécanismes clés du système. Ces extraits sont fournis à titre pédagogique et ne représentent pas le code de production dans son intégralité.

\subsection*{A2.1 – Configuration Spring Security (extrait simplifié)}

L'extrait suivant illustre la configuration de base de Spring Security pour la gestion des filtres JWT et la protection des endpoints.

\begin{lstlisting}[language=Java, caption={Configuration Spring Security pour JWT}, label={lst:spring_security}]
@Configuration
@EnableWebSecurity
public class SecurityConfig {

    @Bean
    public SecurityFilterChain filterChain(HttpSecurity http) 
            throws Exception {
        http
            .csrf(csrf -> csrf.disable())
            .cors(cors -> cors.configurationSource(corsConfig()))
            .sessionManagement(session -> 
                session.sessionCreationPolicy(STATELESS))
            .authorizeHttpRequests(auth -> auth
                .requestMatchers("/api/auth/**").permitAll()
                .requestMatchers("/api/admin/**").hasRole("ADMIN")
                .anyRequest().authenticated())
            .addFilterBefore(jwtAuthFilter, 
                UsernamePasswordAuthenticationFilter.class);
        return http.build();
    }
}
\end{lstlisting}

\subsection*{A2.2 – Structure du payload JWT}

Le jeton JWT généré après authentification complète contient les claims essentiels
présentés ci-dessous.

\begin{samepage}
\begin{lstlisting}[language=Java, caption={Structure du payload JWT}, label={lst:jwt_payload}]
{
  "sub": "user@example.com",
  "userId": 42,
  "accessLevel": "LEVEL_2",
  "mfaCompleted": true,
  "validatedFactors": ["PASSWORD", "OTP"],
  "roles": ["USER"],
  "iat": 1735750000,
  "exp": 1735753600
}
\end{lstlisting}
\end{samepage}

\subsection*{A2.3 – Configuration des exigences MFA par niveau}

L'extrait suivant illustre, sous forme de pseudo-configuration YAML, la correspondance entre les niveaux d'accès et les facteurs MFA requis.

\begin{lstlisting}[language=Java, caption={Exigences MFA par niveau d'accès}, label={lst:mfa_config}]
access-levels:
  LEVEL_1:
    label: "Acces basique"
    required-factors:
      - PASSWORD
      
  LEVEL_2:
    label: "Acces etendu"
    required-factors:
      - PASSWORD
      - OTP
      
  LEVEL_3:
    label: "Acces sensible"
    required-factors:
      - PASSWORD
      - OTP
      - FACE_RECOGNITION
      
  ADMIN:
    label: "Administration"
    required-factors:
      - PASSWORD
      - OTP
      - FACE_RECOGNITION
\end{lstlisting}

% ============================================================================
% ANNEXE A3 : ENDPOINTS PRINCIPAUX
% ============================================================================

\section*{A3 – Endpoints API principaux}
\addcontentsline{toc}{section}{A3 – Endpoints API principaux}

Le tableau \ref{tab:endpoints} récapitule les principaux endpoints exposés par l'API backend.

\begin{table}[H]
\centering
\caption{Endpoints API principaux}
\label{tab:endpoints}
\begin{tabular}{|l|l|l|}
\hline
\textbf{Méthode} & \textbf{Endpoint} & \textbf{Description} \\
\hline
POST & /api/auth/login & Authentification (email + mot de passe) \\
\hline
POST & /api/auth/logout & Déconnexion et invalidation du token \\
\hline
POST & /api/mfa/verify-otp & Validation du code OTP \\
\hline
POST & /api/mfa/verify-face & Vérification biométrique faciale \\
\hline
GET & /api/mfa/status & Statut de la session MFA en cours \\
\hline
POST & /api/otp/setup & Génération du secret OTP (QR code) \\
\hline
POST & /api/otp/confirm & Confirmation de l'activation OTP \\
\hline
GET & /api/users/me & Profil de l'utilisateur connecté \\
\hline
GET & /api/resources & Liste des ressources accessibles \\
\hline
GET & /api/resources/\{id\} & Accès à une ressource spécifique \\
\hline
GET & /api/admin/users & Liste des utilisateurs (admin) \\
\hline
PUT & /api/admin/users/\{id\}/level & Modification du niveau d'accès \\
\hline
GET & /api/admin/logs & Consultation des journaux d'accès \\
\hline
\end{tabular}
\end{table}

% ============================================================================
% ANNEXE A4 : CHECKLIST SÉCURITÉ
% ============================================================================

\section*{A4 – Checklist de conformité sécurité}
\addcontentsline{toc}{section}{A4 – Checklist de conformité sécurité}

Le tableau \ref{tab:checklist_secu} présente une checklist des mesures de sécurité implémentées ou recommandées pour le système.

\begin{table}[H]
\centering
\caption{Checklist de conformité sécurité}
\label{tab:checklist_secu}
\begin{tabular}{|l|c|l|}
\hline
\textbf{Mesure} & \textbf{Statut} & \textbf{Remarque} \\
\hline
Communications HTTPS (TLS 1.2+) & \checkmark & Obligatoire en production \\
\hline
Hachage des mots de passe (bcrypt) & \checkmark & Facteur de coût $\geq$ 10 \\
\hline
Tokens JWT signés (HS256/RS256) & \checkmark & Clé secrète rotative \\
\hline
Expiration des tokens (TTL) & \checkmark & 1h par défaut \\
\hline
Protection CSRF désactivée (API stateless) & \checkmark & Justifié par JWT \\
\hline
Politique CORS restrictive & \checkmark & Origines whitelist \\
\hline
Rate limiting sur /auth/login & \checkmark & 5 tentatives / minute \\
\hline
Verrouillage temporaire après échecs & \checkmark & 15 min après 5 échecs \\
\hline
Journalisation des accès (AccessLog) & \checkmark & Horodatage, IP, résultat \\
\hline
Chiffrement des embeddings biométriques & $\circ$ & Recommandé \\
\hline
Rotation automatique des clés JWT & $\circ$ & À implémenter \\
\hline
Audit de sécurité externe & $\circ$ & Recommandé avant prod \\
\hline
\end{tabular}
\end{table}

\vspace{0.5cm}
\noindent \textbf{Légende :} \checkmark = Implémenté \quad $\circ$ = Recommandé / À implémenter
